\documentclass[11pt]{article}

\usepackage{alltt}
\usepackage{makeidx}
\usepackage[latin1]{inputenc}
\usepackage{url}

\newcommand{\define}[1]{\emph{#1}\index{#1}}
\newcommand{\indextt}[1]{\index{#1@\texttt{#1}}}
\newcommand{\dindextt}[1]{\index{#1@\texttt{#1}|textbf}}

\newcommand{\meta}[1]{\ensuremath{\langle\textit{#1}\rangle}}

\makeindex
%%% makeindex -s pgg-manual.ist pgg-manual.idx
\begin{document}
\title{The PGG System---User Manual}
\author{Peter Thiemann\thanks{Copyright \copyright~Peter Thiemann, 1998-2000}\\
  Universit�t Freiburg\\
  thiemann@informatik.uni-freiburg.de}
\maketitle


\section{Introduction}
\label{sec:introduction}

The PGG system is a partial evaluation system for the full Scheme
language as defined in the R5RS report \cite{KelseyClingerRees1998}.
It has the following features
\begin{itemize}
\item offline partial evaluation using the cogen approach;
\item correct specialization of imperative code;
\item side effects performed at specialization time;
\item modular specialization;
\item no restrictions on primitives and static inputs (they are not
  restricted to have first-order types);
\item handles \texttt{eval}, \texttt{apply},
  \texttt{call-with-values}, and control operators correctly;
\item flexible control of memoization;
\item language extensions (user-defined algebraic datatypes,
  \texttt{make-cell}, \texttt{cell-set!}, \texttt{cell-ref},
  \texttt{cell-eq?}); 
\item representation analysis;
\item fast specialization (the system produces generating extensions);
\item multi-level specialization.
\end{itemize}
The system does not have a post-processor.
This manual does not contain explanatory material about offline
partial evaluation. Section~\ref{sec:partial-evaluation} gives some
pointers to relevant literature.

\begin{small}
\clearpage
\tableofcontents
\clearpage
\end{small}

\section{Installation}
\label{sec:installation}

To use the system, you first have to install the Scheme48 system
\cite{KelseyRees1995}, which is available from Richard Kelsey's homepage
\url{http://www.s48.org/}. The current version 0.57 is required to run PGG.

Once you have installed Scheme48, unpack the distribution file by 
\begin{alltt}
kailua> mkdir pgg-1.3
kailua> cd pgg-1.3 
kailua> zcat \meta{path-where-you-downloaded}/pgg-1.3.tar.gz | tar xvf -
\end{alltt}
(with \verb+kailua>+ being the shell's prompt)
This creates the directory \texttt{pgg-1.3} in the current directory.

Next, you should build yourself an image file of the system, to speed
up loading later on. To do this type:
\begin{verbatim}
kailua> cd pgg-1.3
kailua> make
(echo ",bench on"; \
 echo ",config,load genext-packages.scm pgg-packages.scm"; \
 for package in pgg-residual pgg ; do \
 echo ",load-package $package"; \
 done ; \
 echo ",open pgg signals"; \
 echo ",open auxiliary pgg-library pgg-specialize pp"; \
 echo ",collect"; \
 echo ",dump pgg.image \"(PGG-1.2 made by $LOGNAME `date`)\""; \
 echo ",exit" ) \
| scheme48 -h 10000000
Welcome to Scheme 48 0.57 (made by thiemann on Fri Aug  8 16:50:56 CEST 2003).
Copyright (c) 1993-2001 by Richard Kelsey and Jonathan Rees.
Please report bugs to scheme-48-bugs@martigny.ai.mit.edu.
Type ,? (comma question-mark) for help.
> will compile some calls in line
> > > > > > Before: 2527559 words free in semispace
After:  4203309 words free in semispace
> Writing pgg.image
> 
kailua>
\end{verbatim}

Next time you want to use PGG, type
\begin{verbatim}
kailua> scheme48 -h 6000000 -i pgg.image
\end{verbatim}
to save the time spent with loading and compiling the system. You
might want to put the above into a shell script. The \texttt{-h}
parameter determines the heapsize\index{heapsize} which might have to
be increased when dealing with larger programs. The
\texttt{pgg.image}\dindextt{pgg.image} file may be 
moved to an arbitrary location, it is independent of the directory
containing the PGG distribution.

\section{First Steps}
\label{sec:first-steps}

This section goes through a few examples of using PGG. It assumes that
the system has been started in the \texttt{pgg-1.1} directory. The
subdirectory \texttt{example} contains the sources of all examples.


\subsection{Power}

One of the simplest examples is the exponentiation function \texttt{power}. It
resides in file \texttt{examples/power.scm}. 
\begin{verbatim}
(define (power x n)
  (if (= 0 n)
      1
      (* x (power x (- n 1))))) 
\end{verbatim}
To specialize it, PGG must know three things
\begin{itemize}
\item where to find the source program;
\item the name of the entry point;
\item the binding times of the parameters of the entry point.
\end{itemize}
The latter two are specified using a binding-time skeleton,\index{binding time!skeleton}
i.e., a list that contains the entry point and the binding times of
the parameters. In the example, \texttt{'(power 1 0)} is a sensible
binding-time skeleton. It specifies the entry point \texttt{power} and
the binding times \texttt{1} (dynamic) for the base \texttt{x} and \texttt{0}
(static) for the exponent \texttt{n}.
\indextt{cogen-driver}
\begin{small}
\begin{verbatim}
> (cogen-driver (list "examples/power.scm") '(power 1 0))
bta-run
bta-solve
bta-solve done
'((define (specialize-$goal x2) ...
> 
\end{verbatim}
\end{small}%%$
PGG's answer is the corresponding generating extension. Pretty
printing yields:
\begin{small}
\begin{verbatim}
 (define (specialize-$goal x2)
   (specialize $goal '(power 1 0) (list 'x1 x2)))
 (define (power x_1 n_1)
   (if (_op 0 = 0 n_1)
       (_lift 0 1 1)
       (_op 1 * x_1 (power x_1 (_op 0 - n_1 1)))))
 (define ($goal x_1 n_1)
   (power x_1 n_1))
\end{verbatim}
\end{small}%%$
To use the generating extension, we need to compile it. There are
several ways to do that:
\indextt{cogen-driver}
\begin{small}
\begin{verbatim}
> (define genext
    (cogen-driver (list "examples/power.scm") '(power 1 0)))
bta-run
bta-solve
bta-solve done
; no values returned
> (load-program genext)
; no values returned
>
\end{verbatim}
\end{small}
Alternatively, we can first save the generating extension to a file
and then load and compile the file.
\begin{small}
\begin{verbatim}
> (writelpp genext "/tmp/power-10.scm")
#{Unspecific}
> (load "/tmp/power-10.scm")
/tmp/power-10.scm 
; no values returned
> 
\end{verbatim}
\end{small}
The latter approach is recommended if the source program does not yet
specialize satisfactorily. In this case, inspection of the generating
extension reveals possible problems. For this reason, the syntax of
the generating extension is as close as possible to binding-time
annotated Scheme.

Now that we have loaded the generating extension, we are ready to
specialize. This is facilitated by the \texttt{specialize-\$goal} function %$
provided as part of the generating extension.
\begin{small}
\begin{verbatim}
> (specialize-$goal 0)
'(power-2 x1)
> (get-residual-program)
'((define (power-2 x-3) 1))
> 
\end{verbatim}
\end{small}%%$
The specializer responds with the call template for the residual
program, \texttt{'(power-2 x1)}, indicating that \texttt{power-2}
is the entry point of the residual program and that it takes one
parameter. 
The specializer puts the residual program in a variable
whose contents can be retrieved with the
\indextt{get-residual-program}\texttt{get-residual-program} procedure, for further
examination, compilation, or to save it to a file.

Here is a more interesting run, specializing \texttt{power} for
\texttt{n=4}.
\begin{small}
\begin{verbatim}
> (specialize-$goal 4)
'(power-2 x1)
> (p (get-residual-program))
((define (power-2 x-3)
   (let* ((mlet-11 (* x-3 1))
          (mlet-9 (* x-3 mlet-11))
          (mlet-7 (* x-3 mlet-9)))
     (* x-3 mlet-7))))
\end{verbatim}
\end{small}%%$
(The function \texttt{p} invokes the pretty printer.)

This residual program looks more complicated than we expected. The
reason is that PGG---by default---avoids to duplicate or to reorder
residual code. This feature makes it easy to have impure
(side-effecting) primitives. In the present case, we know that
\texttt{*} is pure and that no code duplication arises from it. 
An appropriate declaration,
\begin{small}
\begin{verbatim}
(define-primitive * - pure)
\end{verbatim}
\end{small}
as provided in the file \texttt{"examples/pure-arith.scm"}, instructs PGG that
\texttt{*} is indeed a pure function.
Now we can say
\begin{small}
\begin{verbatim}
> (define genext
    (cogen-driver (list "examples/power.scm"
                        "examples/pure-arith.scm") '(power 1 0)))
...
> (load-program genext)
; no values returned
> (specialize-$goal 4)
\end{verbatim}
\end{small}%%$
and PGG generates the expected code:
\begin{small}
\begin{verbatim}
(define (power-2 x-3)
  (* x-3 (* x-3 (* x-3 (* x-3 1)))))
\end{verbatim}
\end{small}
A post-processor would have reduced the expression \texttt{(* x-1 1)}
to \texttt{x-1}. This example demonstrates that there is none. It is
nevertheless possible to obtain the same effect by slightly rewriting
the source program. This is left as as exercise.


\subsection{Lambda interpreter}
\label{sec:lambda-interpreter}

This section shows a classic example, an interpreter for an applied
lambda calculus with Scheme's constants,  a conditional, and primitive
operations. The input to the interpreter is a lambda expression, a list of free 
variables, and a list of values of the free variables. The following grammar
specifies the concrete syntax of expressions.
\begin{verbatim}
E ::= X | (lambda (X) E) | (apply E E)
   |  C | (if E E E) | (O E*)
\end{verbatim}
This interpreter employs \emph{partially static data} to represent the
environment. The environment is a list of pairs of variable name and
value. The intention is that the length of the list and all variable
names are static, but the values are dynamic. Traditionally\footnote{In partial 
  evaluation, that is.}, the
Scheme built-in lists cannot be used for this, so we define a new
algebraic datatype for this purpose.
\begin{small}\indextt{define-data}
\begin{verbatim}
(define-data my-list (my-nil) (my-cons my-car my-cdr)) 
\end{verbatim}
\end{small}
This line declares the algebraic datatype \texttt{my-list} with
constructors \texttt{my-nil} and \texttt{my-cons}
(see~\ref{sec:define-data}). The elements of this datatype may be
\define{partially static}, i.e., the components may have a different
(higher) binding time than the structure itself. In addition, they can
be memoized separately. 

It is a little tedious to enter such an environment by hand, so we
also supply a function that transforms a static list of names and a
dynamic list of values into an environment. Finally, it calls the
interpreter function \texttt{int}.
\begin{small}
\begin{verbatim}
(define (main exp names values)
  (let loop ((names names) (values values) (env (my-nil)))
    (if (null? names)
        (int exp env)
        (loop (cdr names) (cdr values)
              (my-cons (my-cons (car names) (car values)) env)))))
\end{verbatim}
\end{small}
The interpreter has two local functions, \texttt{int*} and
\texttt{apply-prim}. \texttt{Int*} evaluates a
list of expressions to a list of values. \texttt{Apply-prim} takes a
primitive operator and a list of value and returns the result. The
interesting part of \texttt{apply-prim} is its use of
\texttt{eval}. \texttt{Eval}'s argument
\texttt{op} is static, whereas the result of \texttt{eval} is
dynamic. 
\begin{small}
\begin{verbatim}
(define (int exp env)
  (let loop ((exp exp))
    (define (int* exp*)
      (let recur ((exp* exp*))
        (if (null? exp*)
            '()
            (cons (loop (car exp*))
                  (recur (cdr exp*))))))
    (define (apply-prim op args)
      (apply (eval op (interaction-environment))
             args))
    (cond
     ((constant? exp)
      exp)
     ((not (pair? exp))
      (lookup exp env))
     ((eq? (car exp) 'IF)
      (let ((test-exp (cadr exp))
            (then-exp (caddr exp))
            (else-exp (cadddr exp)))
        (if (loop test-exp)
            (loop then-exp)
            (loop else-exp))))
     ((eq? (car exp) 'LAMBDA)
      (lambda (y)
        (int (caddr exp) (my-cons (my-cons (caadr exp) y) env))))
     ((eq? (car exp) 'APPLY)
      ((loop (cadr exp))
       (loop (caddr exp))))
     (else
      (apply-prim (car exp) (int* (cdr exp)))))))
\end{verbatim}
\end{small}
All that's missing are two auxiliary functions, \texttt{constant?} and
\texttt{lookup}, that indicate whether an expression denotes a
constant and perform lookup in the environment.
\begin{small}
\begin{verbatim}
(define (constant? e)
  (or (boolean? e)
      (number? e)
      (and (pair? e) (eq? (car e) 'QUOTE))))

(define (lookup v env)
  (let loop ((env env))
    (if (eq? v (my-car (my-car env)))
        (my-cdr (my-car env))
        (loop (my-cdr env)))))
\end{verbatim}
\end{small}
As already mentioned, the idea is that the inputs \texttt{exp}  and
\texttt{names} are static and that \texttt{values} is dynamic. So we start
the binding-time analysis with
\begin{small}
\begin{verbatim}
> (define genext 
    (cogen-driver (list "examples/int.scm") '(main 0 0 1)))
bta-run
bta-solve
bta-solve done
; no values returned
> 
\end{verbatim}
\end{small}
To load this generating extension, we need to load the
\texttt{define-data} operation from module \texttt{pgg-residual}.
\begin{small}\indextt{define-data}
\begin{verbatim}
> ,open pgg-residual
> (load-program genext)
> (specialize-$goal 5 '())
'(main-2 x3)
> (p (get-residual-program))
((define (main-2 x-3) 5))
> (specialize-$goal '(+ x y) '(x y))
'(main-2 x3)
> (p (get-residual-program))
((define (main-2 x-3)
   (let* ((mlet-5 (cdr x-3))
          (mlet-7 (car x-3))
          (mlet-9 (cdr mlet-5))
          (mlet-11 (car mlet-5)))
     (+ mlet-7 mlet-11))))
> (specialize-$goal '(lambda (x) (+ x y)) '(y))
'(main-2 x3)
> (p (get-residual-program))
((define (main-2 x-3)
   (define (loop-4 mlet-3)
     (lambda (y_1-5)
       (+ y_1-5 mlet-3)))
   (let* ((mlet-5 (cdr x-3)) (mlet-7 (car x-3)))
     (loop-4 mlet-7))))
> 
\end{verbatim}
\end{small}%%$
The examples demonstrate that the environment is specialized
away. Only the dynamic values survive and become parameters (this is
called ``arity raising'').
Furthermore, \texttt{eval} and \texttt{apply} have been specialized
satisfactorily, as demonstrated by the last two specializations:
\texttt{(+ mlet-7 mlet-11)} and \texttt{(+ y\_1-5 mlet-3)} is the
corresponding residual code. 

The auxiliary definition of \verb+loop-4+ is introduced automatically by the
specializer to avoid a non-terminating specialization. In the example, there is
no danger of non-termination because the recursive calls only decompose the
source expression. Hence, it is safe to turn off memoization for the function
\texttt{int} by changing the first line of its definition to 
\indextt{define-without-memoization}
\begin{small}
\begin{verbatim}
(define-without-memoization (int exp env)
  ...)
\end{verbatim}
\end{small}
After constructing a new generating extension, we obtain a simpler residual program.
\begin{small}
\begin{verbatim}
(define ($goal-1 values-1)
  (let* ((mlet-2 (cdr values-1))
         (mlet-3 (car values-1)))
    (lambda (y_1-4) (+ y_1-4 mlet-3))))
\end{verbatim}
\end{small}

\subsection{Cyclic}
\label{sec:cyclic}

This example demonstrates specialization of imperative programs.
\begin{small}\indextt{define-data}
\begin{verbatim}
(define-data my-list (my-nil) (my-cons my-car my-cdr)) 
(define (main d)
  (let ((cycle (my-cons 1 (make-cell (my-nil)))))
    (cell-set! (my-cdr cycle) cycle)
    (zip d cycle)))
(define (zip d s)
  (if (null? d)
      '()
      (cons (cons (car d) (my-car s))
            (zip (cdr d) (cell-ref (my-cdr s))))))
\end{verbatim}
\end{small}
The list \texttt{cycle} is
completely static, but the cdr of \texttt{cycle} contains a reference
to cycle itself. This cyclic list of ones is passed as an argument to the
function \texttt{zip} which zips it together with a dynamic
list \texttt{d}. Unrolling the dynamic list involves memoization,
hence the specializer must memoize the cyclic structure passed as an
argument to \texttt{zip} to avoid infinite specialization. Here is
what happens.
\indextt{cogen-driver}
\begin{small}
\begin{verbatim}
> (define genext
    (cogen-driver (list "examples/cyclic.scm") '(main 1)))
bta-run
effect analysis: fixpointing done
bta-solve
bta-solve done
> (p genext)
((define-data my-list (my-nil) (my-cons my-car my-cdr))
 (define (specialize-$goal)
   (specialize $goal '(main 1) (list 'x1)))
 (define (main d_2)
   (let ((cycle_1 (_ctor_memo 0
                              (0 0)
                              #f
                              my-cons
                              1
                              (_make-cell_memo 0
                                               3
                                               0
                                               (_ctor_memo 0
                                                           ()
                                                           #f
                                                           my-nil)))))
     (_message!_memo 0 (_s_t_memo 0 my-cdr cycle_1) cell-set! cycle_1)
     (zip d_2 cycle_1)))
 (define (zip d_1 s_1)
   (multi-memo 1 1 'zip-2 zip-2 #f '(1 0) (list d_1 s_1)))
 (define (zip-2 d_1 s_1)
   (_if 1
        (_op 1 null? d_1)
        (_lift 0 1 '())
        (_op 1
             cons
             (_op 1 cons (_op 1 car d_1) (_lift 0 1 (_s_t_memo 0 my-car s_1)))
             (zip (_op 1 cdr d_1)
                  (_s_t_memo 0 cell-ref (_s_t_memo 0 my-cdr s_1))))))
 (define ($goal d_2)
   (main d_2)))
>
\end{verbatim}
\end{small}
The function \texttt{\_ctor\_memo} constructs the memoized
representation of a constructor. Its first argument is the binding
time of the structure itself, its second argument is the list of
binding times of the components (all \texttt{0} in this
case). \texttt{\_make-cell\_memo} constructs a memoized reference cell,
the first argument is the binding time of the address and the next
argument \texttt{3} is the unique label of the corresponding
\texttt{make-cell} operation in the source
program. \texttt{\_s\_t\_memo} accesses or tests memoized data
objects, the implementation handles them all uniformly.

The operation \texttt{\_define-data} serves to transfer the datatype
definition to the residual program.

To load this generating extension, we need to make the
\texttt{define-data} operation available.
\begin{small}\indextt{define-data}
\begin{verbatim}
> ,open pgg-residual
Load structure pgg-residual (y/n)? y
[pgg-residual
cogen-ctors.scm ]
Newly accessible in user: (define-data)
> (load-program genext)
> (specialize-$goal)
'(main-2 x1)
> (p (get-residual-program))
((define (main-2 x-3)
   (define (zip-4 x-3)
     (let ((mlet-5 (null? x-3)))
       (if mlet-5
           '()
           (let* ((mlet-11 (car x-3))
                  (mlet-9 (cons mlet-11 1))
                  (mlet-13 (cdr x-3))
                  (mlet-15 (zip-4 mlet-13)))
             (cons mlet-9 mlet-15)))))
   (zip-4 x-3)))
\end{verbatim}
\end{small}
The cyclic structure vanishes on specialization. The construction of
the pair \texttt{(x . 1)} is implemented by \texttt{(cons mlet-11 1)}.


\subsection{Guide to the other examples}
\label{sec:guide}

\begin{itemize}
\item \texttt{examples/2lazy.scm} a two-level interpreter for a lazy
  first-order 
  language, implements updatable closures using references.
\begin{small}
\begin{verbatim}
> (define genext
    (cogen-driver (list "examples/2lazy.scm") '(lazy-2int 0 0 0 1)))
> (load-program genext)
\end{verbatim}
\end{small}
The parameters of \texttt{(lazy-2int prg goal xs* xd*)} are
  \begin{itemize}
  \item \texttt{prg} the program;
  \item \texttt{goal} the entry point of \texttt{prg} (a symbol);
  \item \texttt{xs*} the static parameters;
  \item \texttt{xd*} the dynamic parameters.
  \end{itemize}
  The static parameters may include \emph{configuration variables} of
  the form \texttt{(CV $i$)} which refers to the $i$th dynamic
  parameter.

  To perform specialization, we need to load some auxiliary functions
\begin{small}
\begin{verbatim}
(load "examples/2lazy-support.scm")
\end{verbatim}
\end{small}
It contains the example programs \texttt{lazy1} and \texttt{lazy2}.
  Example calls of the specializer include
\begin{small}
\begin{verbatim}
> (specialize $goal '($goal 0 0 0 1) (list lazy1 'f '(42) 'DYN))
> (specialize $goal '($goal 0 0 0 1) (list lazy1 'f '((CV 1)) 'DYN))
> (specialize $goal '($goal 0 0 0 1) (list lazy2 'f '((CV 1) (CV 2) (CV 3)) 'DYN))
> (specialize $goal '($goal 0 0 0 1) (list lazy2 'f '((CV 1) (CV 2) 13) 'DYN))
> (specialize $goal '($goal 0 0 0 1) (list lazy2 'f '((CV 1) 7 11) 'DYN))
> (specialize $goal '($goal 0 0 0 1) (list lazy2 'f '(#t (CV 1) (CV 2)) 'DYN))
> (specialize $goal '($goal 0 0 0 1) (list lazy2 'f '(#f (CV 1) (CV 2)) 'DYN))
> (specialize $goal '($goal 0 0 0 1) (list lazy2 'f '(#f (CV 1) 17) 'DYN))
\end{verbatim}
\end{small}
\item \texttt{examples/app.scm} contains the append function for
  lists.  
\item \texttt{examples/dotprod.scm} compute the scalar product of
  three vectors. This is an example for multi-level
  specialization\index{multi-level specialization}.
  \begin{small}
\begin{verbatim}
> (define genext (cogen-driver (list "examples/dotprod.scm")
                               '(dotprod 0 1 2 3)))
> (load-program genext)
> (specialize-$goal 2)
'(multi-memo 2 2 'dotprod-2 dotprod-2 #f '(0 1 2) (list x2 x3 x4))
\end{verbatim}
    %%$
  \end{small}
  This answer indicates that the residual program is again a
  generating extension, which can be loaded and specialized
  further. Let's have a look
  \begin{small}
\begin{verbatim}
> (p (get-residual-program))
((define (dotprod-2 x-7 x-5 x-3)
   (let* ((mlet-15 (car x-7))
          (mlet-17 (_op 1 car x-5))
          (mlet-13 (_op 1 * (_lift0 1 mlet-15) mlet-17))
          (mlet-19 (_op 2 car x-3))
          (mlet-11 (_op 2 * (_lift 1 1 mlet-13) mlet-19))
          (mlet-21 (cdr x-7))
          (mlet-23 (_op 1 cdr x-5))
          (mlet-25 (_op 2 cdr x-3))
          (mlet-33 (car mlet-21))
          (mlet-35 (_op 1 car mlet-23))
          (mlet-31 (_op 1 * (_lift0 1 mlet-33) mlet-35))
          (mlet-37 (_op 2 car mlet-25))
          (mlet-29 (_op 2 * (_lift 1 1 mlet-31) mlet-37))
          (mlet-39 (cdr mlet-21))
          (mlet-41 (_op 1 cdr mlet-23))
          (mlet-43 (_op 2 cdr mlet-25))
          (mlet-27 (_op 2 + mlet-29 (_lift0 2 0))))
     (_op 2 + mlet-11 mlet-27))))
\end{verbatim}
  \end{small}
  This time, we have to \emph{load the residual program} to continue
  specializing. The answer from the previous specialization step tells
  us the name \texttt{dotprod-2} of the entry point.
  \begin{small}
\begin{verbatim}
> (load-program (get-residual-program))
> (specialize dotprod-2 '(dotprod-2-1 0 1 2) '((111 222) v2 v3))
'(multi-memo 1 1 'dotprod-2-1 dotprod-2-1 #f '(0 1) (list v2 v3))
> (load-program (get-residual-program))
> (specialize dotprod-2-1 '(dotprod-2-1-1 0 1) '((333 444) v3))
'(dotprod-2-1-1 v3)
> (p (get-residual-program))
((define (dotprod-2-1-1 v-3)
   (let* ((mlet-5 (car v-3))
          (mlet-7 (* 36963 mlet-5))
          (mlet-9 (cdr v-3))
          (mlet-11 (car mlet-9))
          (mlet-13 (* 98568 mlet-11))
          (mlet-15 (cdr mlet-9))
          (mlet-17 (+ mlet-13 0)))
     (+ mlet-7 mlet-17))))
\end{verbatim}
  \end{small}
  This is the final specialized program after three steps.
\item \texttt{object} a class of counter objects. A mini-example with
  state.
\begin{small}
\begin{verbatim}
> (define genext (cogen-driver (list "examples/object.scm") '(main)))
> (load-program genext)
> (specialize-$goal)
\end{verbatim}
\end{small}
\item \texttt{pm} Olivier Danvy's pattern matcher \cite{Danvy1991}
  \begin{small}
\begin{verbatim}
> (define genext (cogen-driver (list "examples/pm.scm") '(match 0 1)))
\end{verbatim}
  \end{small}
\item \texttt{unify} imperative unification of terms where variables
  are implemented by references.
  \begin{small}
\begin{verbatim}
> (define genext (cogen-driver (list "examples/unify.scm") '(main 0 1)))
> (load-program genext)
> (specialize-$goal '(cst 555))
> (specialize-$goal '(var 555))
> (specialize-$goal '(bin (var 1) (var 1)))
> (specialize-$goal '(bin (var 1) (bin (cst 4711) (var 1))))
\end{verbatim}
  \end{small}
\end{itemize}


\subsection{Specialization of modular programs}
\label{sec:modular}

As an advanced feature, it is possible to encapsulate the generating extension
in a module. We recap the example of the \texttt{power} function to illustrate
it. In addition to the usual parameters for \texttt{cogen-driver} we need to
specify a filename for the output.
\begin{small}
\begin{verbatim}
> (cogen-driver (list "examples/power.scm") '(power 1 0) "/tmp/power1.scm")
bta-run
bta-solve
bta-solve done
'((define (power x_1 n_1) (if (_op 0 = 0 n_1) (_lift 0 1 1) (_op 1 * x_1 (power x_1 (_op 0 - n_1 1))))) (define ($goal x_1 n_1) (power x_1 n_1)))
>
\end{verbatim}
\end{small}
This command generates two files:
\begin{itemize}
\item \texttt{/tmp/power1.scm} contains the code of the generating extension
  (pretty printed) and
\item \texttt{/tmp/power1.config.scm} contains the declarations for the
  interface and the structure of the generating extension. For the example, PGG 
  generates the following declarations:
  \begin{small}
\begin{verbatim}
(define-interface
  power1-interface
  (export $goal))
(define-structure
  power1
  power1-interface
  (open scheme signals define-data pgg-library)
  (files power1))
\end{verbatim}
  \end{small}
\end{itemize}
To use the generating extension from this module, we need to make Scheme48
aware of it.
\begin{small}
\begin{verbatim}
> ,config,load /tmp/power1.config.scm
/tmp/power1.config.scm
> 
\end{verbatim}
\end{small}
Now the system can load and compile the module, just by referencing it with its 
name.
\begin{small}
\begin{verbatim}
> ,open power1
Load structure power1 (y/n)? y
[define-data cogen-ctors.scm]
[power1 /tmp/power1.scm]
> 
\end{verbatim}
\end{small}
Finally, we can specialize in the same way as before.
\begin{small}
\begin{verbatim}
> (specialize $goal '($goal 1 0) '(x 0))
'($goal-1 x)
> (get-residual-program)
'((define ($goal-1 x-1) 1))
> 
\end{verbatim}
\end{small}
Section~\ref{sec:cogen-driver} in the reference part lists a number of
options to gain more control over the module declaration.


\subsection{Specialization with respect to indexed data}
\label{sec:modular-data}

It is possible to split the static data into an indexed set of data
fragments. The main catch is that only one particular indexed value is
available to each single run of the specializer, the current world. The
specializer can request arbitrary elements (worlds) from this set using a special
construct. If the request concerns the current world then the specializer
continues right away. Otherwise, it checks the memoization cache. If the
requested world has already been seen in the past, it might be possible to
resolve the request. Otherwise, the specializer generates a new memoization
point which waits until the requested world becomes available to the
specializer, possibly for the second time.

The most striking application for this feature is
the separate compilation of modular programs by specializing an interpreter. In 
this application, the index values are the names of modules and the standard
semantics of the special construct is to load the module's text into memory.

An an example, we consider the compilation of a simple register machine
language. Here is an example session.
\begin{verbatim}
> (load "examples/modint-examples.scm")
examples/modint-examples.scm
> (p module1)
((add (jz 1 copy)
      (decr 1)
      (incr 0)
      (jump add))
 (finis))
> (p module2)
((copy (jz 2 test)
       (incr 1)
       (decr 2)
       (jump copy))
 (test (jz 1 finis)
       (jump add)))
\end{verbatim}
The \texttt{main} function of the interpreter for this register-machine
language accepts four parameters, a function that maps a label to a module
name, \texttt{modulename-of}, the  entry label, \texttt{name}, the number of
registers, \texttt{nargs}, and the
initial contents of the registers, \texttt{initial\_args}. The \texttt{name}
and \texttt{nargs}  inputs are known statically, the other inputs are dynamic. 
\begin{verbatim}
> (define genext
    (cogen-driver '("examples/modint-base.scm" "examples/modint.scm")
                  '(main 1 0 0 1)))
bta-run
interpret-type: #(type-all t #(type-app -> (#(type-app b ()) #(type-app -> (#(type-app b ()) #(type-app b ()) #(type-var t))) #(type-var t))))
interpret-type: #(type-all t #(type-var t))
bta-solve
bta-solve done
> ,open pgg-residual
> (writelpp genext "/tmp/modint0.scm")
> (load "/tmp/modint0.scm")
> (specialize-$goal 'add 2)
'(main-1 x1 x4)
\end{verbatim}
Specialization stops right before loading the first module. So far, it
generated code for transferring the input list into the registers:
\begin{verbatim}
> (p (get-residual-program))
((define (main-1 x-2 x-1)
   (let* ((mlet-3 (car x-1))
          (mlet-4 (cdr x-1))
          (mlet-5 (car mlet-4))
          (mlet-6 (cdr mlet-4))
          (mlet-7 (x-2 'add)))
     (jump-global-2 x-2 mlet-3 mlet-5))))
\end{verbatim}
The call to \texttt{jump-global-2} refers to code that will be generated as
soon as the next module becomes available. This fact is signalled to the system
via the 
\texttt{continue} function.\indextt{continue}
\begin{verbatim}
> (continue 'mod1 module1)
\end{verbatim}

At any point between invocations of \texttt{continue} it is possible to suspend 
the state of specialization to a file. The corresponding command is\indextt{suspend}
\begin{verbatim}
> (suspend "/tmp/suspended.scm")
\end{verbatim}

Another, later session with \texttt{pgg} can resume this specialization after
loading the generating extension and reading the suspended file using
\texttt{resurrect}.\indextt{resurrect} 
\begin{verbatim}
> (load "/tmp/modint0.scm")
> (load "examples/modint-examples.scm")
> (resurrect "/tmp/suspended.scm")
#t
> (continue 'mod2 module2)
> (continue 'mod1 module1)
\end{verbatim}

The last two calls to \texttt{continue} complete the specialization of the
interpreter of modular register machine programs.

The file \texttt{modint-mutual.scm} contains a more sophisticated
implementation that compiles each module only once. Here is a transcript:
\begin{verbatim}
> (define genext
    (cogen-driver '("examples/modint-base.scm" "examples/modint-mutual.scm")
                  '(main 0 1 0 1)))
bta-run
interpret-type: #(type-all t #(type-app -> (#(type-app b ()) #(type-app -> (#(type-app b ()) #(type-app b ()) #(type-var t))) #(type-var t))))
interpret-type: #(type-app -> (#(type-app b ()) #(type-app b ())))
bta-solve
bta-solve done
> (writelpp genext "/tmp/regcompiler2.scm")
> (load "/tmp/regcompiler2.scm")
/tmp/regcompiler2.scm
> (specialize-$goal exported-labels 3)
'(main-1 x2 x4)
\end{verbatim}
Here is the startup code for the compiled program:
\begin{verbatim}
> (p (get-residual-program))
((define (main-1 x-2 x-1)
   (let* ((mlet-3 (car x-1))
          (mlet-4 (cdr x-1))
          (mlet-5 (car mlet-4))
          (mlet-6 (cdr mlet-4))
          (mlet-7 (car mlet-6))
          (mlet-8 (cdr mlet-6)))
     (case x-2
       ((add) (jump-2 mlet-3 mlet-5 mlet-7))
       ((finis) (jump-3 mlet-3 mlet-5 mlet-7))
       ((copy) (jump-4 mlet-3 mlet-5 mlet-7))
       (else (dyn-error "Unknown name"))))))
\end{verbatim}
Here is the code for the first module:
\begin{verbatim}
> (continue 'mod1 module1)
> (p (get-residual-program))
((define (jump-2 mlet-3 mlet-2 mlet-1)
   (if (zero? mlet-2)
       (jump-4 mlet-3 mlet-2 mlet-1)
       (jump-2 (+ mlet-3 1) (- mlet-2 1) mlet-1)))
 (define (jump-3 mlet-3 mlet-2 mlet-1)
   mlet-3))
\end{verbatim}
Here is the code for the second module:
\begin{verbatim}
> (continue 'mod2 module2)
> (p (get-residual-program))
((define (jump-5 mlet-3 mlet-2 mlet-1)
   (if (zero? mlet-2)
       (jump-3 mlet-3 mlet-2 mlet-1)
       (jump-2 mlet-3 mlet-2 mlet-1)))
 (define (jump-4 mlet-3 mlet-2 mlet-1)
   (if (zero? mlet-1)
       (jump-5 mlet-3 mlet-2 mlet-1)
       (jump-4 mlet-3 (+ mlet-2 1) (- mlet-1 1)))))
\end{verbatim}

The input for this section, along with one more example, can be found in file
\texttt{examples/sample\_modules\_session.scm}. 

\clearpage
\section{Reference manual}
\label{sec:reference}


\subsection{Notation}
\label{sec:notation}

The syntax definition uses an extended BNF where all symbols are
terminals, except 
\begin{itemize}
\item nonterminal symbols are capitalized;
\item \verb-::=-, \verb-|-, \verb-*-, \verb-+-, \verb-[-, \verb-]- are
  metasymbols with the usual meaning (definition, alternative, zero or
  more repetitions, one or more repetitions, begin of optional part, end
  of optional part).
\end{itemize}

\subsection{Type system}
\label{sec:type-system}

The type system is the system of simple types with a $\top$ type and
recursion. The type language comprises the types
\begin{itemize}
\item basic, for every expression that \emph{never} evaluates to a
  function or an element of an algebraic datatype as defined by
  \texttt{define-data};\indextt{define-data}
\item $[\tau_1,\dots,\tau_n]\rightarrow\tau_0$, for every expression that
  \emph{always} evaluates to a function;
\item $\texttt{TC}[\tau_1,\dots,\tau_n]$, for every expression that
  \emph{always} evaluates to an element of the algebraic datatype
  named \texttt{TC} (the $\tau_i$ are the types of the arguments of
  the constructors in some unspecified fixed order);
\item $\top$, for every expression that cannot be given one of the
  other types, in particular for an expression that may evaluate
  \begin{itemize}
  \item to a function and also to some non-function value;
  \item to an element of datatype \texttt{TC} and also to an element
    of a different datatype \texttt{TC'} or an element of a
    non-algebraic type.
  \end{itemize}
\end{itemize}
The PGG system performs type inference for this system using
Henglein's algorithm \cite{Henglein1991}.
Due to the presence of recursive types, the result of the type
inference is a graph where each node is annotated with a type constructor.

\subsection{Binding-time analysis}
\label{sec:binding-time-analysis}

The binding-time analysis assigns a binding-time to each node in the
type graph and ensures that the binding-time assignment is
well-formed. Wellformedness of such a binding-time assignment $B$
means that the annotation on a type node 
is always less than or equal to the annotations on the direct
descendants of that node. A node of type $\top$ is well-formed if it
assumes the maximum possible binding time (it is kept dynamic
throughout all stages of specialization). That means that potential
type clashes are postponed to the last stage of running the
program. 

The binding-time analysis inserts
a lift-expression on top of each expression of basic type that occurs
as
\begin{itemize}
\item the argument of a primitive operations;
\item the argument of a function;
\item the ``then'' or ``else'' arm of a conditional; or
\item the argument of a data constructor.
\end{itemize}
A lift-expression injects a value computed at specialization time into
the residual program.


\subsection{Primitive operations}
\label{sec:primitive-operations}

For primitive operations, the binding time analysis imposes a second
set $S$ of binding-time annotations on the nodes of the type graph. The
well-formedness criterium for them has two aspects. First, the $S$
annotation is always greater than or equal to the $B$
annotation. Second, the $S$ annotation of a node is greater than or
equal to the $S$ annotations of each direct descendant node in the
type graph. For each expression that performs a primitive operation,
the specializer requires that all arguments are \emph{leveled}. That is,
the binding time analysis enforces that the $S$ and $B$ annotations of
all arguments are equal. In consequence, all argument computations
take place at the \emph{same} binding time (see
\cite{Thiemann1996-reflection}). 

This restriction makes it safe to allow primitive operations to have
functions as arguments. 

\subsection{Representation analysis}
\label{sec:representation-analysis}

PGG performs a representation analysis that assigns to each node in
the type graph yet another binding time $M$. The $M$ annotation of a
type is wired in such a way that it reflects the maximum level $+1$ at which
expressions of that type will be subject to memoization, $0$ if they
are never memoized. For example,
an expression of a type with an $M$ value of $0$ will never be 
memoized. All those expressions will use the standard representation
of values of that type. If the $B$ annotation is zero and the $M$
value is greater than zero, the memoized representation will be used,
which incurs a runtime overhead. The general condition is that
expressions with $B<M$ use the memoized representation and the others
use the standard representation.

Unfortunately, there is a catch: if a type is memoized and at the same
time required to be leveled then the type must assume the maximum
binding time. This is, because primitive operations cannot deal with
the memoized representation. The catch is that we now have a cyclic
dependency: $B$ implies the placement of memoization points, which
implies a setting of $M$, which implies a deteriorated setting of $B$,
which implies the placement of more memoization points, and so on.

\subsection{Memoization}
\label{sec:memoization}

\index{memoization point}%
PGG automatically inserts memoization points on top of dynamic
conditionals and on top of dynamic lambdas. However, this only happens
if the branches of the dynamic conditional or the body of the dynamic
lambda contains a control transfer at specialization time, i.e., a
static function call. Furthermore, if several of these are nested in
an expression, only the outermost receives a memoization point. This
is a slight refinement of the standard strategy
\cite{BondorfDanvy1991,Bondorf1991}. 

It is possible to turn this feature off for a given function by
defining it using \texttt{define-without-memoization}
(see~\ref{sec:define-without-memoization}). In any case, 
memoization points can be defined and inserted manually
(see~\ref{sec:define-memo}). 



\subsection{Special expressions}
\label{sec:special-operations}


\subsubsection{eval}\dindextt{eval}

The binding-time analysis and the specializer treat \texttt{eval}
specially. The binding-time analysis enforces that the argument of
\texttt{eval} is leveled. The result type is also leveled and it may
have any binding time that is greater than or equal to the argument's
binding time. 

If the binding times are equal then the \texttt{eval} function is
called at the specified level. If the binding times differ by one then
the specializer simply drops the static argument value into the
residual program. Otherwise, the specializer preserves the static
argument for the next level and decrements the binding-time
difference.


\subsubsection{apply}\dindextt{apply}

If a function is declared as the \texttt{apply} function
(see~\ref{sec:define-primitive}) then the specializer uses a special
postprocessor that transforms expressions of the form
\begin{verbatim}
(apply f (cons x1 (cons x2 (cons x3 ...))))
\end{verbatim}
into
\begin{verbatim}
(f x1 x2 x3 ...)
\end{verbatim}
To this end, it is necessary to declare \texttt{cons} as \texttt{pure}\indextt{pure}
(see~\ref{sec:define-primitive}). 

\subsubsection{lambda-poly}\dindextt{lambda-poly}

In addition to the standard lambda abstraction there is a polyvariant memoizing 
abstraction operator.
\begin{verbatim}
E ::= (lambda-poly (V*) E*)
\end{verbatim}
The specializer treats a dynamic \texttt{lambda-poly} just like an ordinary
\texttt{lambda}. A static \texttt{lambda-poly} specializes to a vector of all
required specializations of the abstraction. The specializer constructs a
partially static value consisting of a memoization map and a reference to
the vector. Both, the vector and the memoization map, are initially
empty. Whenever the specializer applies a \texttt{lambda-poly} it looks up the
static skeleton of the arguments in the memoization map. If a specialization
for this skeleton is already present in the map, it constructs a reference to
the corresponding position in the vector. Otherwise, it extends the memoization 
map, constructs a new specialization of the \texttt{lambda-poly}, and inserts
it into the vector.

This construct implements a first-class memoization mechanism and it could be
used to replace the usual memoization. 

\subsection{Predefined operators}
\label{sec:extra}

These operators can be used in source programs.

The module \texttt{cogen-boxops} exports the following operators that
manipulate references (boxed values):
\dindextt{make-cell}\dindextt{cell-ref}\dindextt{cell-set"!}
\begin{verbatim}
(make-cell exp)
\end{verbatim}
allocates a new mutable reference cell which initially contains the
value of \texttt{exp}. Returns the reference to the new cell.
\begin{verbatim}
(cell-ref exp)
\end{verbatim}
returns the value stored in the referenced cell 
if \texttt{exp} evaluates to a reference.
\begin{verbatim}
(cell-set! exp1 exp2)
\end{verbatim}
stores the value of \texttt{exp2} in the cell referenced by
\texttt{exp1} provided this value is a reference. The return value is
unspecified.  

\subsection{Directives}
\label{sec:directives}

The directives are only allowed at the top-level of a source program.

\subsubsection{define-without-memoization}
\dindextt{define-without-memoization}
\label{sec:define-without-memoization}

\begin{verbatim}
D ::= (define-without-memoization (P V*) D0* E*)
  |   (define-without-memoization P E)
\end{verbatim}
Define procedure \texttt{P}. The specializer does not to automatically
insert memoization points in the body of \texttt{P}.


\subsubsection{define-data}
\dindextt{define-data}
\label{sec:define-data}

\begin{verbatim}
D ::= (define-data TC (C Ci*)+)
  |   (define-data TC hidden (C Ci*)+)
\end{verbatim}
Define the algebraic datatype \texttt{TC} with constructors \texttt{C}
and selectors \texttt{Ci}. In addition, constructor test operations \texttt{C?}
are defined. The name \texttt{TC} is used during type checking to
check for equality of types.

For example, the declaration
\begin{verbatim}
(define-data list
  (nil)
  (cons car cdr))
\end{verbatim}
defines the constructors \texttt{nil} (nullary) and \texttt{cons}
binary, the constructors tests \texttt{nil?} and \texttt{cons?}, and
the selectors \texttt{car} and \texttt{cdr}.

\index{partially static}
The binding-time analysis considers algebraic datatypes as partially
static, i.e., the arguments of a constructor can have a different
(higher) binding time than the constructor itself. In such cases, the
specializer performs arity raising when appropriate. The constructors,
selectors, and test operations are binding-time polyvariant, i.e.,
each use of such an operation may have a different binding-time
pattern.  

The second form of \texttt{define-data} declares a datatype whose elements are
ignored by the memoization mechanism. This is a potentially dangerous  
feature because it can change the meaning of a program during specialization by
cheating the memoization mechanism.

\subsubsection{define-type}
\dindextt{define-type}
\begin{verbatim}
D ::= (define-type (P B*) B)
\end{verbatim}
Declares the arity of primitive operation \texttt{P}. The actual
values of the \texttt{B}s are currently ignored.

Example:
\begin{verbatim}
(define-type (cons * *) *)
\end{verbatim}
declares the operator \texttt{cons} of arity 2.

A variable which is declared as an
operator but does not occur at operator position in an expression
is eta-expanded by the frontend according to its declaration.


\subsubsection{define-primitive}
\dindextt{define-primitive}
\label{sec:define-primitive}

\begin{verbatim}
D ::= (define-primitive O T [dynamic|error|opaque|pure|apply|Number])
\end{verbatim}
\dindextt{dynamic}\dindextt{error}\dindextt{opaque}\dindextt{pure}\dindextt{apply}
Declares the operator \texttt{O}  of type \texttt{T} with an optional
property.

The parameter \texttt{T} declares the type of the operator.  
It can be either \texttt{-}, indicating that the type of
\texttt{O} is not restricted, or it can specify a polymorphic type for
\texttt{O}. The grammar is as follows:
\begin{verbatim}
T  ::=  - | T0
T0 ::=  (all TV T0) | (rec TV T0) | (TC T0*) | TV
\end{verbatim}
Here, \texttt{TV} stands for a type variable (an arbitrary symbol) and
\texttt{TC} stands for a type constructor (an arbitrary symbol). The syntax,
\texttt{(all TV T0)}, declares that variable \texttt{TV} is all-quantified in
\texttt{T0}, like $\forall alpha.\tau_0$. The syntax, \texttt{(rec TV T0)},
declares a recursive type, like $\mu\alpha.\tau_0$. The remaining cases are
type constructor application and occurrence of a type variable. For
convenience, the function type constructor, \texttt{->}, is treated
specially. Writing \texttt{(-> t1 ... tn t0)} declares a Scheme function that
takes \texttt{n} parameters ($\texttt{n} \ge 0$) and delivers a result of type
\texttt{t0}. 
 
The properties \texttt{dynamic} and \texttt{opaque} are synonyms. Each
of them forces the binding time of \texttt{O}'s result to be dynamic. The
property \texttt{error} advises the binding-time analysis that the result of the
operation can assume any type whatsoever (because an error primitive raises an
exception and never returns a value) and that its binding time is determined
from the binding times of the arguments as with any primitive operation.
The remaining properties advise the specializer what to do when it
residualizes the operator.
The \texttt{pure} property states that the operator does not have side
effects. Instead of creating a let-binding for the expression
\texttt{(O V*)}, the specializer will treat it as a value, potentially
discarding or duplicating the expression.
The \texttt{apply} property states that the operator \texttt{O} is the
\texttt{apply} function as defined in the Scheme standard.

If the property is a \texttt{Number} it declares the least binding time of the
operator. 
\index{binding time!specification of}


\subsubsection{define-memo}
\dindextt{define-memo}
\label{sec:define-memo}

\begin{verbatim}
D ::= (define-memo M Number [Active])
  |   (define-memo M Number 'deferred)
\end{verbatim}
\index{memoization point}%
Defines \texttt{M} as a unary operator indicating a memoization point at
level \texttt{Number}. This is also the binding time of the memoization point.
For two-level specialization this number is
\texttt{1}. Useful in connection with
\texttt{define-without-memoization}.\indextt{define-without-memoization}
The optional parameter \texttt{Active} defines the minimum level of
specialization at which the specialization point is active. The default is
\texttt{0}, that is, the specialization point is always active. Useful in
connection with multi-level specialization if the same program is specialized
with different levels.\index{multi-level specialization}

Both parameters, \texttt{Number} and \texttt{Active} may be integer expressions 
using the free variable \texttt{max-level}, which is bound to the maximum
binding-time presently in use. 

Example: Similix defines the operator \texttt{\_sim-memoize} as an indicator
for memoization points. To achieve the same 
behavior in PGG requires the following declaration.
\begin{verbatim}
(define-memo _sim-memoize 1)
\end{verbatim}

The second form of the directive declares an operator to construct deferred
memoization points. An applied occurrence of a deferred memoization point has
the form
\begin{verbatim}
(M V E)
\end{verbatim}
\index{memoization point!deferred}
When specialization hits upon a deferred memoization point, it extracts the
static skeleton and looks it up in a secondary cache. Just as with standard
memoization points, it creates a function call to a specialized version of
\texttt{E}. The difference is that the specialization of \texttt{E} depends on
a future value of \texttt{V}, so it must be deferred until the value of
\texttt{V} becomes available.

\subsubsection{load}
\dindextt{load}

\begin{verbatim}
D ::= (load Filename)
\end{verbatim}
includes the contents of the file named by the string
\texttt{Filename} into the program. The included file may contain
additional \texttt{load}s, without limit on the nesting level. The
\texttt{Filename} argument is always interpreted relative to the
current directory that Scheme48 is running in.


\subsubsection{begin}
\label{sec:begin}
\dindextt{begin}

\begin{verbatim}
D ::= (begin D*)
\end{verbatim}
As in Scheme, top-level definitions may appear nested inside a
\texttt{begin}. This is handy if a macro is to expand to more than one
definition. 


\subsection{Commands}
\label{sec:top-level}

This section summarizes the available top-level commands of the PGG system.

\subsubsection{Creating a generating extension}
\label{sec:cogen-driver}
\dindextt{cogen-driver}

The main entry point of the system is the function \texttt{cogen-driver}. It
takes the following parameters
\index{binding time!skeleton|textbf}
\begin{verbatim}
(cogen-driver InputSpec BindingTimeSkeleton)
\end{verbatim}
where
\begin{itemize}
\item \texttt{InputSpec} is either a single string specifying the name
  of a ``jobfile'' that contains a list of filenames, or a list of
  strings each of which specifies a filename: 
  the source program is the content of all these files concatenated together.
\item \texttt{BindingTimeSkeleton} is a list where the first element is a
  symbol denoting the main function of the source program and the remaining 
  elements are binding times for the parameters of the main
  function. The main function must have exactly as many parameters as the
  \texttt{BindingTimeSkeleton} indicates.
\end{itemize}
\index{binding time!specification of}
A binding time is a non-negative integer, with \texttt{0}
denoting static. The PGG system assumes that the binding time of the result of
the main function is the maximum of \texttt{1} and the binding times of the
function's parameters, i.e., the \texttt{BindingTimeSkeleton}.

The result of calling \texttt{cogen-driver} is a generating extension, i.e., a
list of Scheme definitions that can be loaded and run. 

This function is defined in module \texttt{pgg}. If it is not accessible try
\begin{verbatim}
> ,open pgg
\end{verbatim}
at the top-level Scheme48 prompt.

\index{specialization!modular programs}%
A number of optional parameters may be specified after the
\texttt{BindingTime\-Skeleton} argument. They allow the generation of the
generating extension as a Scheme48 module. The following options are
recognized.
\begin{itemize}
\item \texttt{(goal SYMBOL)} specifies that \texttt{SYMBOL} will be
  the name of the specialization entry point (i.e., the first
  parameter to \texttt{specialize}) of the generating extension.
\item \texttt{(export SYMBOL ...)} adds the listed \texttt{SYMBOL}s to the
  export list of the generated module.
\item \texttt{(open SYMBOL ...)} considers the listed \texttt{SYMBOL}s as
  module names that are to be opened to run the generating extension.
\item \texttt{(files SYMBOL ...)} considers the listed \texttt{SYMBOL}s as
  names of files that will be included in the generating module.
\item \texttt{(SYMBOL ...)} is included as an option line in the structure
  declaration for the generating module.
\item \texttt{STRING} the name of the file where the module should be
  written. Must be the last; subsequent options are ignored. Writes the files
  \texttt{STRING.scm} and \texttt{STRING.config.scm}, after stripping any
  extension from \texttt{STRING}.
\end{itemize}

\subsubsection{Running a generating extension}
\label{sec:specialize}
\dindextt{specialize}

The function \texttt{specialize} is the human interface to running a generating
extension. It has two forms.
\begin{verbatim}
(specialize GenextMain BindingTimeSkeleton ARGS)
\end{verbatim}
and
\begin{verbatim}
(specialize BindingTimeSkeleton ARGS NEW-GOAL)
\end{verbatim}
where
\index{binding time!skeleton|textbf}
\begin{itemize}
\item \texttt{GenextMain} is the main function of the generating extension,
\item \texttt{BindingTimeSkeleton} is a list where the first element
  is a symbol denoting the \emph{name} of main function of the
  generating extension, and the remaining elements are binding times
  its parameters. The list of binding times must be identical to the
  one given to \texttt{cogen-driver} when creating this generating
  extension.
\item \texttt{ARGS} is the list of arguments to the generating extension. Its
  length must be equal to the number of binding times supplied in the
  \texttt{Binding\-Time\-Skeleton}. The 
  positions corresponding to \texttt{0} entries in the skeleton contain the
  currently static arguments. The positions corresponding to other entries in
  the skeleton must contain symbols, they are used as stubs for generating
  identifiers. 
\item (optional argument) \texttt{NEW-GOAL} is the name of the entry point for
  the specialized program. If it is not specified, PGG invents one for you, but 
  admittedly not a very original one.
\end{itemize}
The result is a call template for the specialized function, a list
consisting of the name of the function and of names of the
arguments. The residual program can be retrieved via
\texttt{(get-residual-program)}\dindextt{get-residual-program}. It is a list
of Scheme definitions. Furthermore, the variable
\texttt{*support-code*}\dindextt{*support-code*} contains additional
code, for example data 
definitions that are necessary to run the specialized program.

If the returned call template has the form
\begin{verbatim}
'(multi-memo Level 'Goal Goal Bts Args)
\end{verbatim}
then we have done one step of a \define{multi-level specialization}
\cite{GlueckJoergensen1995-plilp}. It means that
the residual program is again a generation extension where 
\texttt{Level} is a number, \texttt{Goal} is the name of the 
entry point, \texttt{Bts} are the binding times of the arguments, and
\texttt{Args} is a list of symbols of the same length.
It can be loaded as usual and specialized again with
\texttt{specialize} by constructing the
\texttt{BindingTimeSkeleton} from \texttt{Goal} and \texttt{Bts}.


This function is defined in module \texttt{pgg-residual}. If it is not
accessible try
\begin{verbatim}
> ,open pgg-residual
\end{verbatim}
at the top-level Scheme48 prompt.


\subsubsection{Continuing a specialization}
\label{sec:continue}
\dindextt{continue}

The function \texttt{continue} continues a specialization that has been
suspended at a deferred memoization point.
\begin{verbatim}
(continue Name Arg)
\end{verbatim}
The parameter \texttt{Name} identifies the index that the specialization waits
for. It is used to match the index value in pending deferred memoization
points. The example in Section~\ref{sec:modular-data} uses the symbols
\texttt{'mod1} and \texttt{'mod2} for this purpose.

The parameter \texttt{Arg} is the indexed value. This is a value of base type
and its representation is completely up to the
programmer. The example in Section~\ref{sec:modular-data} uses the empty list
to indicate an empty world.


\subsubsection{Suspend a deferred specialization}
\label{sec:suspend}
\dindextt{suspend}

The function \texttt{suspend} writes the current memoization cache and the
cache of deferred specializations to a file.
\begin{verbatim}
(suspend Filename)
\end{verbatim}
The \texttt{Filename} parameter indicates the name of the file in which the
memoization cache and the deferred cache are stored.

\subsubsection{Resurrect a deferred specialization}
\label{sec:resurrect}
\dindextt{resume}

The function \texttt{resurrect} installs a memoization cache and deferred cache 
from a file. It sets up the system to continue a previously suspended
specialization.
\begin{verbatim}
(resurrect Filename)
\end{verbatim}
The \texttt{Filename} parameter indicates the name of the file in which the
memoization cache and the deferred cache are stored.

Returns \verb!#t! if the file was successfully read. Otherwise, it returns
\texttt{\#f}. 

\subsection{Settable options}
\label{sec:options}

These options are accessible in module \texttt{cogen-globals} except where
otherwise noted. Some of
them only make sense for programmers who want to use the frontend
separately. 

\begin{itemize}
\item \verb+(set-bta-display-level! n)+ Default: $1$.
  \dindextt{set-bta-display-level"!}
  
  Display output from the
  binding-time analysis, $0 \le \texttt{n} \le 4$.  $0$
  means no output.
\item \verb+(set-effect-display-level! n)+  Default: $1$.
  \dindextt{set-effect-display-level"!}

  Display output from the
  effect analysis. $0$ means no output.
\item \verb+(set-scheme->abssyn-let-insertion! v)+ Default:
  \texttt{\#f}.
  \dindextt{set-scheme->abssyn-let-insertion"!}

  Instruct the
  frontend to insert let expressions for the bound variables of
  lambdas and definitions. 

  Useful if the frontend is to be used in other projects.
\item \verb+(set-memo-optimize! v)+ Default: \texttt{\#t}.
  \dindextt{set-memo-optimize"!}

  Optimize the representation of functions and algebraic
  datatypes. Use expensive memoized representation only for data that
  actually passes a memoization point.
\item \verb+(set-abssyn-maybe-coerce! v)+ Default: \texttt{\#t}.
  \dindextt{set-abssyn-maybe-coerce"!}

  Instructs the frontend to insert provisional lift expressions at
  certain places. The backend eliminates these later on if they are
  useless. Can be turned of for using the frontend separately.
\item \verb+(set-generate-flat-program! b)+ Default: \texttt{\#f}.
  \dindextt{set-generate-flat-program"!}

  Instructs the generating extension to produce flat programs. By default, the
  residual programs have exaclty one top-level definition, all others are
  nested inside and invisible to the outside.
\item \verb+(gensym-ignore-name-stubs!)+ (module \texttt{cogen-gensym})
  \dindextt{gensym-ignore-name-stubs"!}

  Instructs the generating extension to ignore name stubs when generating fresh 
  symbols.
\item \verb+(gensym-use-name-stubs!)+ (module \texttt{cogen-gensym}) Default.
  \dindextt{gensym-use-name-stubs"!}

  Instructs the generating extension to use provided name stubs wherever
  possible. 
\item \verb+(set-memolist-stages! n)+ Default: \texttt{0}.
  \dindextt{set-memolist-stages"!}

  Set optimization level for memoization table. If set to \texttt{n} then
  memoization uses \texttt{n} cascaded association lists, indexed by the first
  \texttt{n} elements of the static projection at a memoization point.
\item \verb+(set-lambda-is-pure! v)+ Default: \texttt{\#t}.
  \dindextt{set-lambda-is-pure"!}

  The code generator considers lambda abstractions as pure values if this flag
  is set.
\item \verb+(set-lambda-is-toplevel! v)+ Default: \texttt{\#f}.
  \dindextt{set-lambda-is-toplevel"!}

  Generate a toplevel function for each memoized lambda abstraction if
  set. Required for \texttt{suspend}\indextt{suspend} and
  \texttt{resurrect}\indextt{resurrect} to work properly. 

\end{itemize}


\subsection{Utilities}
\label{sec:utilities}

\dindextt{p}
Pretty printing is available through function \texttt{p} in module
\texttt{pretty-print}. The function takes one parameter, the
expression that is to be pretty-printed.

\dindextt{writelpp}
The function \texttt{(writelpp LIST FILE)} writes the list
\texttt{LIST} to file \texttt{FILE} applying the pretty printer to
each element of the list. It is defined in module \texttt{auxiliary}.

The contents of modules are generally available by typing
\begin{small}
\begin{verbatim}
> ,open Modulename
\end{verbatim}
\end{small}
to the top-level command line interpreter of Scheme48.

\section{Differences to Scheme}
\label{sec:differences}

PGG assumes a declarative semantics: the order of a sequence of
top-level definitions does not matter, even if they are spread over
several files.

PGG implements R5RS macros with the restriction that macros defined by
\texttt{let-syntax} and \texttt{letrec-syntax} cannot expand to macro
definitions.


\section{Reading a generating extension}
\label{sec:reading}

For debugging purposes it is sometimes helpful to read the generating
extension, because it is a representation of the binding-time
annotated program. Besides standard Scheme constructs it contains the
following kinds of expressions, most of which are implemented as
macros in module \texttt{pgg-library}. The semantics of the ellipsis
\texttt{...} is the same as in the \texttt{syntax-rules} patterns of
Scheme \cite{KelseyClingerRees1998}, zero or more repetitions of the
preceding item. The list is sorted alphabetically.

\begin{itemize}
\item \verb!(multi-memo level fname fct bts args)! denotes a
  memoization point at level \texttt{level}, \texttt{fname} is a
  symbol specifying the name of the generating function to run,
  \texttt{fct} is the function itself, \texttt{bts} is a list of
  binding times describing the arguments of the function,
  \texttt{args} is the list of arguments (must have the same length as
  \texttt{bts})
\item \verb!(_app lv f a ...)! application of non-memoized function
  \texttt{f} to arguments \texttt{a ...} at level \texttt{lv}
\item \verb!(_app lv f a ...)! application of memoized function
  \texttt{f} to arguments \texttt{a ...} at level \texttt{lv}
\item \verb!(_begin lv bl e1 e2)! a \texttt{begin} at level
  \texttt{lv}, \texttt{bl} is the binding time of \texttt{e1}
\item \verb+(_cell-set!_memo lv ref arg)+ updates a memoized reference
  cell \texttt{ref} at level \texttt{lv} with value \texttt{arg}
\item \verb!(_cell-eq?_memo lv ref1 ref2)! tests two memoized
  reference cells \texttt{ref1} and \texttt{ref2} for equality at
  level \texttt{lv}
\item \verb!(_ctor_memo lv (bt ...) ctor arg ...)! creates a memoized
  object with constructor \texttt{ctor}, \texttt{lv} is the binding
  time of the structure, \texttt{(bt ...)} are the binding times of
  the arguments \texttt{arg ...}
\item \verb!(_eval lv diff body)! the \texttt{body}  becomes available
  at level \texttt{lv} then it is delayed for \texttt{diff} levels
\item \verb!(_freevar lv arg)! a free variable \texttt{arg} at level
  \texttt{lv}
\item \verb!(_if lv bl e1 e2 e3)! the conditional at level
  \texttt{lv}, \texttt{bl} is the binding time of the branches
  \texttt{e2} and \texttt{e3}, \texttt{e1} is the condition
\item \verb!(_lift0 lv val)! delay the value \texttt{val} to level
  \texttt{lv}
\item \verb!(_lift lv diff value)! the \texttt{value} becomes
  available at level \texttt{lv}, then it is delayed for another
  \texttt{diff} levels
\item \verb!(_lambda lv v* e)! non-memoized lambda abstraction at
  level \texttt{lv}, formals \texttt{v*}, body \texttt{e}
\item \verb!(_lambda_memo lv arity label fvs bts f)! memoized lambda
  abstraction at level \texttt{lv}, \texttt{arity} is a list of
  symbols serving as stubs for variable names, \texttt{label} is the
  unique label of the lambda, \texttt{fvs} is a list of  the (values
  of the) free variables, \texttt{f} is a function that maps the
  values of the free variables and the variable names generated from
  \texttt{arity} to a new body
\item \verb!(_make-cell_memo lv lab bt arg)! creates a memoized
  reference cell at level \texttt{lv}, with unique label \texttt{lab},
  \texttt{bt} is the binding time of the argument \texttt{arg}
\item \verb!(_op lv op arg ...)! the operator \texttt{op} applied to
  \texttt{arg ...} at level \texttt{lv}
\item \verb!(_op_pure lv op arg ...)! the pure operator \texttt{op}
  applied to \texttt{arg ...} at level \texttt{lv}
\item \verb!(_s_t_memo lv sel v)! a selector or test for a memoized
  datastructure at level \texttt{lv}, \texttt{sel} is the selector or
  test function, \texttt{v} is the argument
\item \verb!(_vlambda lv (fixed-var ...) var body)! same as
  \verb!_lambda!, but for variable arity; the list of
  \texttt{fixed-var} names the obligatory arguments and \texttt{var}
  names the optional argument list
\item \verb!(_vlambda_memo lv fixed-vars var label vvs bts f)! memoized lambda
  abstraction with variable  arity functions, see
  \verb!_lambda_memo! and \verb!_vlambda! for explanation
\end{itemize}

\section{Technical background}
\label{sec:background}


\subsection{Partial evaluation in general}
\label{sec:partial-evaluation}


Good starting points for the study of partial evaluation are
Jones, Gomard, and Sestoft's textbook~\cite{JonesGomardSestoft1993},
Consel and Danvy's tutorial notes~\cite{ConselDanvy1993}, Mogensen and
Sestoft's encyclopedia chapter~\cite{MogensenSestoft1997},
and Gallagher's tutorial notes on partial
deduction~\cite{Gallagher1993}. Further material can be found in
the proceedings of the Gammel Avern{\ae}s
meeting (PEMC)~\cite{PEMC1988,PEMC1988-ngc}, in the
proceedings of the ACM conferences and workshops on Partial Evaluation and
Semantics-Based Program Manipulation
(PEPM)~\cite{PEPM1991,PEPM1992,PEPM1993,PEPM1994,PEPM1995,PEPM1997,PEPM1999},
and in special issues of various
journals~\cite{jfpspecial1993,jlpspecial1993,lascspecial1995,tcsspecial1998}. A
comprehensive volume on partial evaluation appeared in the Lecture
Notes of Computer Science series~\cite{PE1996}.  Sestoft maintains
an online bibliography~\cite{pebibliography}.

The above paragraph is taken from the introduction to the 1998
Symposium on Partial 
Evaluation \cite{SOPE1998} which is a collection of concise
articles characterizing the state of the art, stating challenging problems,
and outlining promising directions for future work in partial evaluation.


\subsection{Directly related publications}
\label{sec:publications}

The following publications explain various parts of the PGG system.
\begin{itemize}
\item Cogen in Six Lines \cite{Thiemann1996-icfp} explains how to
  derive a handwritten multi-level cogen from a multi-level
  specializer and applies this to the construction of a
  continuation-based handwritten multi-level cogen. This is done in
  continuation-passing style and in direct style with control
  operators.
\item\indextt{eval}\indextt{apply}
  Towards Specialization of Full Scheme
  \cite{Thiemann1996-reflection} explains the specialization of 
  \texttt{eval}, \texttt{apply}, and \texttt{call/cc}. It relies on a
  binding-time analysis that allows for higher-order primitive
  operations.
\item Implementing Memoization for Partial Evaluation
  \cite{Thiemann1996-plilp} gives details about the implementation
  strategy for partially static values in PGG.
\item Correctness of a Region-Based Binding-Time Analysis
  \cite{Thiemann1997-mfps} defines and proves correct a binding-time
  analysis for a lambda calculus with side-effects.
\item Sound Specialization in the Presence of Computational Effects
  \cite{LawallThiemann1997} defines a specialization calculus based on
  Moggi's computational lambda calculus and shows how to implement
  it. This calculus is the basis of PGG's specialization algorithm.
\end{itemize}


\subsection{Structure of the implementation}
\label{sec:structure}

The frontend of PGG is similar to the frontend of a Scheme compiler.

The first pass renames all variables, expands macros, expands
backquotes, transforms named lets to letrecs, and collects mutable
variables.
\indextt{cell-set"!}\indextt{cell-ref}
The second pass performs assignment conversion, eliminating all
\texttt{(set! v e)} operations in favor of \texttt{(cell-set! v e')},
replacing all uses of \texttt{v} by \texttt{(cell-ref v)}, and changing
the definition of \texttt{v} accordingly.
The third pass performs lambda lifting.
The fourth pass transforms to abstract syntax and performs eta
expansion.

The next phase is binding-time analysis. It consists of type
inference, effect analysis (if \texttt{cell-set!} and friends have
been used), construction of the binding-time constraints, solution of
the constraints, and the introduction of memoization points.

Finally, the backend produces the generation extension.

\section{Known problems}
\label{sec:problems}

\begin{itemize}
\item Syntax errors are not dealt with gracefully.
\item Do not use identifiers that end with
  \verb!([-_][0-9]+)+! (interpreted as a regular expression for the
  \texttt{regexp} library) for procedures and global variables.
\end{itemize}



\subsection*{Acknowledgments}

Most parts of the system have been developed while the author was at
T�bingen University.
Special thanks to Michael Sperber for unwaveringly testing the system,
pushing it to its limits, suggesting new features, finding many
problems (as well as some 
surprising features), and supplying some bug fixes. Thanks also to
Simon Helsen and Frank Knoll who suffered through various versions of
the system. 

\bibliography{abbrevs,papers,collections,misc,books}
\bibliographystyle{plain}
\printindex
\end{document}
